% introduction
Automatic Speech Recognition (ASR) has many practical applications nowadays, e.g. in dictation systems for medical documentation and journalism.
Another application comes from the rapidly increasing amount of videos available online on video platforms for entertainment and learning, such as Youtube, Vimeo, Coursera or OpenHPI.
All of these benefit from automatically generated transcripts and subtitles.
But, the result of many ASR systems is an unformatted, long text without punctuation marks, such as periods and commas.
These texts are hard to read, understand, and use without manually inserting the missing punctuation marks.

In this paper we want to address this problem by automatically creating punctuated text from unpunctuated text.
We use the concept of deep neural networks to process the unformatted transcript as well as the speech as input.
We train both models independently and retrieve their separate predictions.
Afterwards the results are combined in a fusion step.
The final output can replace the original output from ASR systems and improve readability and quality of transcripts.
Additionally, the puctuation marks often represent suitable boundaries for subtitles, enhancing their overall quality.

The rest of the paper is structured as follows:
Related work is summarized in Section~\ref{sec:related_work}.
Section~\ref{sec:training_data} describeds the datasets we use for training and evaluation.
The data preprocessing, training, and evaluation of our lexical and our acoustic model can be seen in Section~\ref{sec:lexical_model} and Section~\ref{sec:acoustic_model} respectively.
Details of the fusion step are explained in Section~\ref{sec:fusion}.
We show our demo application in Section~\ref{sec:demo} and conclude our work in Section~\ref{sec:future}.
