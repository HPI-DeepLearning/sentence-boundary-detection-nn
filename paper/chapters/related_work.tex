As punctuation prediction is a mandatory preprocessing step for further working with automated speech recognition output, a lot of research has been done in this field.
Some approaches focus only on the lexical part~\cite{Gravano2009, Lu2010, Ueffing2013, Cho2012, Zhang2013}.
Gravano et al.~\cite{Gravano2009} used a text-based n-gram language model to detect punctuation (comma, period, question mark).
Dynamic conditional random fields are used by Lu and Ng~\cite{Lu2010} and Ueffing et al.~\cite{Ueffing2013}.
Ueffing et al. evaluate their method with different features, such as language model scores, parse trees, dynamic sentence length and token n-grams.
The usefulness of the individual features highly depends on the nature of the text, which is processed.
For example, if the text is well structured, the parse tree features improve the result.
Similar to our approach, Cho et al.~\cite{Cho2012} use a sliding window over the input data to predict different punctuations.
Zhang et al.~\cite{Zhang2013} predict punctuations of an input stream.
For each processed word in the input stream, syntactic features are used to predict the punctuation symbol after that word.
The used features include, e.g., part-of-speech tags, tree-based features (the parse tree is build step by step) or bag of words.

Most of the researchers combine prosodic features, such as pitch, pauses, duration, and lexical features, such as words, n-grams, part-of-speech tags~\cite{Mark1999, Christensen2001, Liu2005, Matusov2007, Wang2012}.
Chen~\cite{Mark1999} predict punctuation on the basis of prosodic features in a first step using a Hidden Markov Model.
In a second step they use a language model to adapt the predicted punctuation from the first step.
Christensen et al.~\cite{Christensen2001} focus on multi-layer perceptron methods to combine prosodic and lexical features, whereas Liu et al.~\cite{Liu2005} use conditional random fields.
Matusov et al.~\cite{Matusov2007} optimize their approach to the needs of machine translation.
They combine a language model and prosodic features in a log-linear model and add a phrase coverage feature, which is motivated by phrase-based machine translation systems.
A comparison of different machine learning models to combine prosodic and lexical feature for the prediction of punctuation was done by Wang et al.~\cite{Wang2012}.
The dynamic conditional random fields achieve the best result on a broadcast news copora (F1-Measure of 42.8\%).

In this paper we present a new approach: predicting punctuations using deep learning.
To the best of our knowledge, such an approach was never tried before.
We learn two individual models, one based on lexical features and the other one based on prosodic features.
In the end, the predictions of both models are fused.