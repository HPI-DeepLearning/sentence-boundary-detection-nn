% how did we prepare our data

Our data set can be separated into to parts:
The first part is based on TED talks\footnote{\url{ted.com}}, the second part is a subset from Wikipedia\footnote{\url{en.wikipedia.org}}.

\paragraph{TED talks} Our TED talk data set consists of TODO talks. For each talk we have the following data files:
\begin{itemize}
	\item .xml-file: This file contains a manually created script of the talk.
	The text is formatted and serves as training data for the lexical model.
	\item .ctm-file: This is a time-marked speech input.
	It contains one word per line with the second the word was said in the talk and its duration.
	Additionally, each sentence is labeled in the file, so that the data can be used for training of the acoustic model.
	\item .sph-file: This file contains pulse code modulation (PCM) data.
	This data can be converted into .wav files.
\end{itemize}

\paragraph{Wikipedia} We used a small subset of the English Wikipedia to obtain more training data.
We selected only those articles with more than 10,000 characters.
Besides, we extracted the plain text of the articles, discarding lists, headlines, tables, etc.
In total we received around 3.5 million new training instances from the Wikipedia articles.
